\documentclass{article}
\usepackage[a4paper, hmargin=1.4in, vmargin=1in]{geometry}
\usepackage{graphicx}

\title{\LARGE\textbf{Repertorios de Instrucciones: \\ 
                     Carater\'{i}sticas y Funciones}}
\author{\large\textit{Daniel Araya Román}}
\date{}

\begin{document}
\maketitle

\subsection*{Elementos de una instrucci\'{o}n m\'{a}quina}
Los elementos esenciales de las instrucciones de los computadores son,
el c\'{o}digo de operaci\'{o}n \textbf{(codop)}. Referencias a operandos
de origen y destino y referencia a la siguiente instrucci\'{o}n.
Los operandos de origen, son valores de entrada para la operaci\'{o}n
que se va a realizar. Los operandos de destino, son valores de salida
de la operaci\'{o}n que se va a realizar.

Es importante destacar que los operandos de origen y destino pueden ser
almacenados en memoria principal o virtual, registros de la CPU o en
dispositivos de entrada/salida.

\begin{figure}[h]
    \centering
    \includegraphics[width=0.9\textwidth]{instruction_cycle.png}
    \caption{Ciclo de una instrucci\'{o}n.
    \cite {stallings2006organización}}
\end{figure}

\subsection*{Tipos de las Instrucciones}
Un lenguaje de alto nivel, expresa las operaciones de forma concisa.
Mientras que un lenguaje m\'{a}quina expresa las operaciones de una manera
elemental. Es fundamental que el repertorio de instrucciones m\'{a}quina
sea lo suficientemente rico para que se puedan expresar las operaciones
de un lenguaje de alto nivel. Los tipos de instrucciones se pueden clasificar en:

\begin{itemize}
    \item \textbf{Transferencia de Datos:}
          instrucciones de entrada y salida de datos.
    \item \textbf{Almacenamiento de Datos:}
          instrucciones de memoria.
    \item \textbf{Procesamiento de Datos:}
          instrucciones aritm\'{e}ticas y l\'{o}gicas.
    \item \textbf{Control de Flujo}:
          instrucciones de comprobaci\'{o}n y de bifurcaci\'{o}n.
\end{itemize}

\subsection*{Representaci\'{o}n de las Instrucciones}
Dentro del computador cada instrucci\'{o}n se representa mediante una
secuencia de bits. Dividida en campos correspondientes a los elementos
que la constituyen. En la mayor\'{i}a de repertorios de instrucciones
se utiliza m\'{a}s de un formato de instrucci\'{o}n. Durante la ejecuci\'{o}n
del programa, la instrucci\'{o}n se almacena en el registro de instrucci\'{o}n
(IR) de la CPU. La CPU extrae los campos de la instrucci\'{o}n y los
utiliza para realizar la operaci\'{o}n correspondiente. Los c\'{o}digos de
operaci\'{o}n (codop) se representan mediante abreviaturas denominadas
\textbf{mnem\'{o}nicos}.

\begin{figure}[h]
    \centering
    \includegraphics[width=0.7\textwidth]{instruction_format.png}
    \caption{Formato de una instrucci\'{o}n.
    \cite {stallings2006organización}}
\end{figure}

\subsection*{N\'{u}mero de direcciones}
La mayor\'{i}a de las instrucciones de los computadores actuales son
una, dos o tres direcciones. Las instrucciones de tres direcciones
no son comunes ya que requieren formatos relativamente largos.

\begin{figure}[h]
    \centering
    \vspace{0.25cm}
    \includegraphics[width=0.8\textwidth]{instruction_addressing.png}
    \caption{N\'{u}mero de direcciones.
    \cite {stallings2006organización}}
\end{figure}

\begin{flushleft}
    La instrucci\'{o}n de una sola direcci\'{o}n es muy simple. Para que funcione,
    una segunda direcci\'{o}n debe de estar impl\'{i}cita. Esto era usual en las
    primeras m\'{a}quinas. La direcci\'{o}n impl\'{i}cita era un registro conocido
    como el \textbf{acumulador (AC)}.
\end{flushleft}
Incluso es posible que una instrucci\'{o}n no tenga direcciones. Esto
es logra utilizando la pila. El n\'{u}mero de direcciones es una decisi\'{o}n
b\'{a}sica de dise\~{n}o. Menos direcciones significa instrucciones m\'{a}s primarias,
lo que requiere un procesador menos complejo. Por otro lado, un mayor n\'{u}mero de
direcciones significa instrucciones m\'{a}s complejas, pero menos instrucciones para
realizar una tarea.

\section*{Dise\~{n}o del Repertorio de Instrucciones}
Es importante entender que \textbf{el repertorio de instrucciones es el medio que tiene
    el programador para poder controlar el procesador.} En consecuencia debe de adaptarse
a las necesidades del programador. Los aspectos fundamentales del dise\~{n}o del
repertorio de instrucciones son.

\begin{itemize}
    \item \textbf{Repertorio de Operaciones:}
          El repertorio de operaciones es el conjunto de operaciones que se
          pueden realizar en el procesador. C\'{u}antas y cu\'{a}n complejas deben ser.
    \item \textbf{Tipos de Datos:}
          Los distintos tipos de datos que se pueden utilizar en las operaciones.
    \item \textbf{Formatos de Instrucciones:}
          Longitud de la instrucci\'{o}n (en bits), n\'{u}mero de direcciones,
          tama\~{n}o de los campos, etc.
    \item \textbf{Registros:} n\'{u}mero de registros del procesador que se
          pueden referenciar por las instrucciones y el uso de ellos.
    \item \textbf{Modos de direccionamiento:}
          Los distintos modos de direccionamiento que puede especificarse la
          direcci\'{o}n de un operando.
\end{itemize}

\subsection*{Tipos de operandos}
Los operandos de una instrucci\'{o}n pueden ser de distintos tipos tales como:
\textbf{direcciones}, \textbf{n\'{u}meros}, \textbf{caracteres} o
\textbf{datos l\'{o}gicos}.

\subsection*{Tipos de operaciones}
Las operaciones que se pueden realizar en un procesador se pueden clasificar
en la siguiente forma:

\begin{itemize}
    \item \textbf{Transferencia de datos:}
          Mueven datos entre registros y memoria. Deben de especificarse
          la direcci\'{o}n de origen y la direcci\'{o}n de destino. Que
          podr\'{i}an ser de memoria, un registro, cabecera de la pila o
          un dispositivo de entrada/salida. Indicarse la longitud de datos
          a transferir. Especificar el modo de direccionamiento.

    \item \textbf{Operaciones aritm\'{e}ticas:}
          Realizan operaciones aritm\'{e}ticas sobre datos num\'{e}ricos.
          Suma, resta, multiplicaci\'{o}n, divisi\'{o}n, etc.

    \item \textbf{Operaciones l\'{o}gicas:}
          Realizan operaciones l\'{o}gicas sobre datos l\'{o}gicos.
          Manipulaci\'{o}n de bits, operaciones booleanas, desplazamiento,
          rotaci\'{o}n, etc.

    \item \textbf{Operaciones de conversi\'{o}n:}
          Arquellas que cambian el formato de los datos. Conversi\'{o}n
          binario a decimal, decimal a binario, etc.

    \item \textbf{Operaciones de control del sistema:}
          Realizan llamadas al sistema operativo, interrupciones.
          Instrucciones privilegiadas, que solo pueden ser ejecutadas
          por el sistema operativo.

    \item \textbf{Operaciones de control de programa:}
          Cambiar la secuencia de ejecuci\'{o}n de las instrucciones.
          Saltos, bifurcaciones, llamadas a subrutinas, etc.
\end{itemize}

\subsection*{Instrucciones de bifurcaci\'{o}n}
Las instrucciones de bifurcaci\'{o}n son aquellas que cambian la secuencia
de ejecuci\'{o}n de las instrucciones. Las instrucciones de bifurcaci\'{o}n
pueden ser condicionales o incondicionales. Por ejemplo:

\begin{verbatim}
                    is_zero proc
                    again:
                        mov ax, 0
                        je end          ; conditional jump
                        jmp again       ; unconditional jump
                    
                    end:
                        ret
                    is_zero endp
\end{verbatim}

\subsection*{Instrucciones de llamada a procesamiento}
Las razones principales para utilizar procedimientos son la econom\'{i}a
y la modularidad. Ya que un procedimiento permite que la misma porci\'{o}n
de c\'{o}digo se puede reutilizar muchas veces. Tamb\'{i}en estos permiten
que programas largos, se puedan subdividir en unidades m\'{a}s peque\~{n}as.

\begin{figure}[h]
    \begin{verbatim}
                    ; GetMousePosition
                    ;
                    ; int 33H
                    ; ax = 03H
                    ; bx: button status
                    ; cx: x position
                    ; dx: y position
                    ;
                    GetMousePosition proc far
                    mov ax, 03H
                    int 33H
                    mov [mouseX], cx
                    mov [mouseY], dx
                    mov [mouseStatus], bx
                    ret
                    GetMousePosition endp
                    
                    ; SetMousePosition
                    ;
                    ; int 10H
                    ; ax = 02H
                    ; bx = 00H
                    ; dh = x position
                    ; dl = y position
                    ;
                    SetMousePosition proc far
                    mov ah, 02H
                    mov bh, 00H
                    int 10H
                    ret
                    SetMousePosition endp
    \end{verbatim}
    \caption{Ejemplo de procedimientos en ensamblador.}
\end{figure}

Es importante entender que el uso de procedimientos requiere de 
dos instrucciones b\'{a}sicas, \textbf{call}, que produce la bifurcaci\'{o}n.
\textbf{ret}, instrucci\'{o}n de retorno. El CPU debe de guardar la direcci\'{o}n
de retorno, puede estar normalmente en tres lugares, un registro, la pila o
la principio del procedimiento. Solamente que tiene el limitante que no puede
hacer uso de procedimientos reentrantes, en otras palabras. \textbf{Procedimientos 
recursivos.}

\begin{figure}[h]
    \centering
    \includegraphics[width=1\textwidth]{stack_procedures.png}
    \caption{Uso de la pila.
    \cite {stallings2006organización}}
\end{figure}

\subsection*{C\'{o}digos de condici\'{o}n}
Son bits de reigstros especiales que pueden ser activados por ciertas
operaciones, y ser utilizados en bifurcaciones condicionales.

\begin{figure}[h]
    \centering
    \includegraphics[width=1\textwidth]{condition_codes.png}
    \caption{C\'{o}digos de condici\'{o}n.
    \cite {stallings2006organización}}
\end{figure}

\bibliography{stallings}
\bibliographystyle{plain}

\end{document}