\documentclass{article}
\usepackage[a4paper, hmargin=1.4in, vmargin=1in]{geometry}
\usepackage{graphicx}

\title{\LARGE\textbf{El Computador}}
\author{\large\textit{Daniel Araya Román}}
\date{}

\begin{document}
\maketitle

\section*{Arquitectura de Von Neumann}
Los computadores actuales se han basado en el dise\~{n}o de la
arquitectura de Von Neumann, la cual se caracteriza por tener una
estructura de almacenamiento de datos y programas en una misma memoria,
y una unidad de procesamiento que ejecuta instrucciones almacenadas en
la memoria. Esta arquitectura se puede ver en la figura 1.

\subsection*{Hardware y Software}
Para poder crear un c\'{a}lculo concreto se necesita de una configuraci\'{o}n
de componentes l\'{o}gicos dise\~{n}ados para dicho c\'{a}lculo. Como si se
tratase del proceso de conexi\'{o}n de diversos componentes para as\'{i}
obtener la configuraci\'{o}n deseada. A esta configuraci\'{o}n se le conoce
vagamente como \textbf{Hardware}. El cu\'{a}l realiza funciones respecto a las
distintas se\~{n}ales de control que recibe.

En contraparte, el conjunto de instrucciones que se le dan se\~{n}ales de control
al hardware para que realice una tarea espec\'{i}fica se denomina como
\textbf{Software}. Estas instrucciones se pueden almacenar en la memoria del
computador, y se pueden modificar para as\'{i} obtener distintos resultados.

\begin{figure}[h]
    \centering
    \includegraphics[width=1\textwidth]{von_neumann.png}
    \caption{Arquitectura de Von Neumann.}
\end{figure}

Por lo tanto, se puede deducir que la funci\'{o}n b\'{a}sica de un computador es
la ejecuci\'{o}n de un programa, el cu\'{a}l se encuentra en un conjunto de
instrucciones almacenadas en la memoria.
\newpage

\section*{Componentes del Computador}

\subsection*{Unidad Central de Procesamiento (CPU)}
El objetivo principal del procesador es ejecutar las instrucciones almacenadas en
la memoria. Est\'{a} compuesta por la Unidad Aritm\'{e}tico L\'{o}gica (ALU),
el Contador de Programa (PC), el Registro de Instrucci\'{o}n (IR) y la Unidad de
Control (CU), los registros de prop\'{o}sito general, el Registro de Direcci\'{o}n
de Memoria (MAR), el Registro de Datos de Memoria (MDR) y el Acumulador.

\subsubsection*{Unidad Aritmético Lógica (ALU)}
La ALU es la encargada de realizar las operaciones aritm\'{e}ticas y l\'{o}gicas
de los datos que se encuentran en el acumulador y en los registros de prop\'{o}sito
general. Estas operaciones se realizan mediante se\~{n}ales de control que se
encuentran en la Unidad de Control (CU).

\subsubsection*{Contador de Programa (PC)}
El PC es un registro que contiene la direcci\'{o}n de la siguiente instrucci\'{o}n
a ejecutar. Esta direcci\'{o}n se incrementa en una unidad cada vez que se ejecuta
una instrucci\'{o}n.

\subsubsection*{Registro de Instrucción (IR)}
El IR es un registro que contiene la instrucci\'{o}n que se est\'{a} ejecutando en
ese momento. Esta instrucci\'{o}n se obtiene de la memoria principal, y se almacena
en el IR para que la CU pueda decodificarla y as\'{i} ejecutarla.

\subsubsection*{Unidad de Control (CU)}
La CU es la encargada de decodificar las instrucciones que se encuentran en el IR,
y as\'{i} enviar las se\~{n}ales de control necesarias para que la ALU pueda realizar
las operaciones aritm\'{e}ticas y l\'{o}gicas, y para que el PC pueda incrementar
su valor en una unidad.

\subsubsection*{Registro de Dirección de Memoria (MAR)}
El MAR es un registro que contiene la direcci\'{o}n de memoria de la instrucci\'{o}n
que se encuentra en el IR. Esta direcci\'{o}n se obtiene de la memoria principal, y
se almacena en el MAR para que la CU pueda decodificarla y as\'{i} ejecutarla.

\subsubsection*{Registro de Datos de Memoria (MDR)}
El MDR es un registro que contiene el dato que se encuentra en la direcci\'{o}n de
memoria que se encuentra en el MAR. Este dato se obtiene de la memoria principal, y
se almacena en el MDR para que la CU pueda decodificarla y as\'{i} ejecutarla.

\subsubsection*{Registros de propósito general}
Los registros de prop\'{o}sito general son registros que contienen datos que se
utilizan para realizar operaciones aritm\'{e}ticas y l\'{o}gicas. Estos registros
se pueden utilizar para almacenar datos que se encuentran en la memoria principal,
o para almacenar datos que se encuentran en el acumulador.

\subsubsection*{Acumulador}
El acumulador es un registro que contiene datos que se utilizan para realizar
operaciones aritm\'{e}ticas y l\'{o}gicas. Este registro se puede utilizar para
almacenar datos que se encuentran en la memoria principal, o para almacenar datos
que se encuentran en los registros de prop\'{o}sito general.
\newpage

\section*{Funcionamiento del Computador}
La ejeuci\'{o}n m\'{a}s b\'{a}sica de un computador se puede resumir en los siguientes
pasos:

\subsection*{Ciclo de instrucci\'{o}n}
\begin{enumerate}

    \item \textbf{Captaci\'{o}n de la instrucci\'{o}n}:
          El procesador obtiene la instrucci\'{o}n desde la memoria principal
          utilizando la dirección almacenada en el contador de programa (PC).

    \item \textbf{Decodificaci\'{o}n de la instrucci\'{o}n}:
          El procesador interpreta y decodifica la instrucción obtenida,
          identificando la operación a realizar y los operandos involucrados.

    \item \textbf{C\'{a}lculo de la direcci\'{o}n de memoria del operando}:
          Se determina la dirección de memoria del operando, que puede ser necesario
          para instrucciones que involucren acceso a datos en memoria.

    \item \textbf{Captaci\'{o}n de operando}:
          El procesador accede a la memoria para obtener los datos necesarios seg\'{u}n la
          direcci\'{o}n calculada en el paso anterior y los coloca en registros internos,
          como el Registro de Datos de Memoria (MDR).

    \item \textbf{Ejecuci\'{o}n de la instrucci\'{o}n}:
          El procesador realiza la operaci\'{o}n aritm\'{e}tica o l\'{o}gica que se
          encuentra en el acumulador y en los registros de prop\'{o}sito general.

    \item \textbf{Almacenamiento del resultado}:
          El resultado de la operación se almacena, ya sea en registros internos o de nuevo
          en la memoria, dependiendo de la naturaleza de la operaci\'{o}n y la arquitectura
          del procesador.
    \item \textbf{Termina la ejecuci\'{o}n de la instrucci\'{o}n}:
          El procesador incrementa el valor del contador de programa (PC) en una unidad,
          para as\'{i} poder ejecutar la siguiente instrucci\'{o}n.
\end{enumerate}

\begin{figure}[h]
    \centering
    \includegraphics[width=1\textwidth]{instruction_cycle.png}
    \caption{Ciclo de instrucci\'{o}n.}
\end{figure}

\newpage

\section*{Interrupciones}
El procesador suspende temporalmente la ejecuci\'{o}n de un programa para atender una
solicitud de servicio externo. Carga un nuevo valor en el contador de programa (PC)
para que la ejecuci\'{o}n del programa de comienzo en una rutina de servicio de
interrupci\'{o}n. Una vez que se ha atendido la solicitud, el procesador carga el valor
del contador de programa (PC) que se encontraba antes de la interrupci\'{o}n, y as\'{i}
contin\'{u}a la ejecuci\'{o}n del programa.

\begin{figure}[h]
    \centering
    \includegraphics[width=1\textwidth]{interrupt.png}
    \caption{Ciclo de instrucci\'{o}n con interrupciones.
    \cite{stallings2006organización}}
\end{figure}

\bibliography{stallings}
\bibliographystyle{plain}

\end{document}