\documentclass{article}
\usepackage[a4paper, hmargin=1.4in, vmargin=1in]{geometry}
\usepackage{multicol}

\title{\LARGE\textbf{Registros y Segmentos}}
\author{\large\textit{Daniel Araya Román}}
\date{}
\begin{document}
\maketitle

\section*{Visi\'{o}n General}
Las estructuras de datos que se utilizan en ensamblador
corresponden a las estructuras f\'{i}sicas de los registros
de la memoria principal. Sus instrucciones tienen una relaci\'{o}n
directa y un\'{i}voca con las intrucciones del lenguaje m\'{a}quina.
\medbreak

\begin{multicols}{2}
    \subsection*{\hspace*{1cm} Ventajas}
    \begin{itemize}
        \item \textbf{Velocidad}: Los registros son los dispositivos
              de almacenamiento directamente del procesador.

        \item \textbf{Flexibilidad}: Los registros pueden ser utilizados
              para almacenar cualquier tipo de dato.

        \item \textbf{Rutinas Optimizadas}: Las rutinas de ensamblador
              naturalmente est\'{a}n optimizadas para trabajar con registros.

        \item \textbf{Control absoluto del PC}: El programador tiene
              control absoluto sobre el PC, lo que permite optimizar el
              flujo de ejecuci\'{o}n del programa.
    \end{itemize}
    \columnbreak

    \subsection*{\hspace*{1cm} Desventajas}
    \begin{itemize}
        \item \textbf{Tiempo de programaci\'{o}n}: El tiempo de programaci\'{o}n
              es mayor que en lenguajes de alto nivel.

        \item \textbf{C\'{o}digo fuente extenso}: El c\'{o}digo fuente es
              extenso y dif\'{i}cil de leer.

        \item \textbf{Dependencia de la arquitectura}: El c\'{o}digo fuente
              depende de la arquitectura del procesador, por lo que no es
              portable.

        \item \textbf{Afectar recursos del sistema}: El programador debe
              tener cuidado de no afectar recursos del sistema.
    \end{itemize}
\end{multicols}

\section*{Segmentos}
Es un bloque de memoria designado a un programa, que puede estar ubicado
en cualquier parte de la memoria principal. S\'{o}lo necesita tanto espacio
como el programa requiera para su ejecuci\'{o}n. Se puede tener cualquier
n\'{u}mero de segmentos en memoria. Los segmentos principales son, \textbf{el segmento
de c\'{o}digo, el segmento de datos y el segmento de pila.}

\subsection*{Segmento de C\'{o}digo}
El \textbf{\textit{registro CS}}, direcciona este segmento, por medio de la primera instrucci\'{o}n
ejecutable del programa, que se encuentra al inicio del mismo. El segmento de c\'{o}digo
contiene las instrucciones para la ejecuci\'{o}n del programa. El sistema operativo enlaza esa
localidad, para iniciar la ejecuci\'{o}n del programa.

\subsection*{Segmento de Datos}
El \textbf{\textit{registro DS}}, direcciona este segmento, que contiene las variables y constantes
del programa. El segmento de datos es de lectura y escritura, por lo que se pueden modificar los
valores de las variables y constantes.

\subsection*{Segmento de Pila}
El \textbf{\textit{registro SS}}, direcciona este segmento, almacena las direcciones y los valores
de manera temporal, para el uso de las rutinas del programa. El segmento de pila es de lectura y
escritura, por lo que se pueden modificar los valores de las direcciones y los valores de la pila.

\section*{Registros del Procesador}
\begin{multicols}{2}
    \subsection*{Registros de Segmento}
    \begin{itemize}
        \item \textbf{CS}: \textit{Code Segment}
        \item \textbf{DS}: \textit{Data Segment}.
        \item \textbf{SS}: \textit{Stack Segment}.
        \item \textbf{ES}: \textit{Extra Segment (ES)}.
        \item \textbf{FS}: \textit{Extra Segment (FS)}.
        \item \textbf{GS}: \textit{Extra Segment (GS)}.
    \end{itemize}
    \vphantom{}
    \columnbreak

    \subsection*{Registros de Prop\'{o}sito General}
    \begin{itemize}
        \item \textbf{AX}: \textit{Accumulator}.
        \item \textbf{BX}: \textit{Base}.
        \item \textbf{CX}: \textit{Counter}.
        \item \textbf{DX}: \textit{Data}.
    \end{itemize}
    \vphantom{}
\end{multicols}

\begin{multicols}{2}
    \subsection*{Registros apuntadores de \'{i}ndice}
    \begin{itemize}
        \item \textbf{SI}: \textit{Source Index}.
        \item \textbf{DI}: \textit{Destination Index}.
    \end{itemize}
    \vphantom{}
    \columnbreak

    \subsection*{Registros apuntadores de pila}
    \begin{itemize}
        \item \textbf{SP}: \textit{Stack Pointer}.
        \item \textbf{BP}: \textit{Base Pointer}.
    \end{itemize}
    \vphantom{}
\end{multicols}

\subsection*{Registro CS}
\textit{\textbf{DOS}} almacena la direcci\'{o}n de la primera instrucci\'{o}n ejecutable
del programa en el registro \textbf{CS}. Junto al registro \textbf{IP, registro apuntador
de instrucci\'{o}n}, indica la direcci\'{o}n de la siguiente instrucci\'{o}n a ser buscada
para su ejecuci\'{o}n.

\subsection*{Registro DS}
\textit{\textbf{DOS}} almacena la direcci\'{o}n del segmento de datos en el registro \textbf{DS}.
El registro \textbf{DS} direcciona el segmento de datos, que contiene las variables y constantes
del programa. M\'{a}s un valor de desplazamiento, direcciona una referencia a un
\textit{byte (8-bit)} en espec\'{i}fico dentro del segmento de datos.

\subsection*{Registro SS}
\textit{\textbf{DOS}} almacena la direcci\'{o}n del segmento de pila en el registro \textbf{SS}.
El registro \textbf{SS} direcciona el segmento de pila, m\'{a}s un valor de desplazamiento del
registro \textbf{SP}, direcciona una referencia a un \textit{word (16-bit)} en espec\'{i}fico
dentro del segmento de pila que est\'{a} siendo direccionado.

\subsection*{Registro IP}
\textit{\textbf{DOS}} almacena la direcci\'{o}n de la siguiente instrucci\'{o}n a ser ejecutada
en el registro \textbf{IP}. Est\'{a} asociado al registro \textbf{CS}, el IP indica la direcci\'{o}n
de la instrucci\'{o}n actual en el segmento de c\'{o}digo, que se est\'{a} ejecutando.

\subsection*{Registro apuntadores de \'{i}ndice}
Los registros \textbf{SI} y \textbf{DI} son registros de \'{i}ndice de 16 bits, que se utilizan
para direccionamiento indexado. Ambos se pueden realizar operaciones de cadenas de caracteres,
el registro \textbf{SI}  est\'{a} relacionado con el registro \textbf{DS}, mientras que el
registro \textbf{DI} est\'{a} relacionado con el registro \textbf{ES}.

\subsection*{Registro Generales}
Los registros \textbf{AX}, \textbf{BX}, \textbf{CX} y \textbf{DX} son registros de prop\'{o}sito
general de 16 bits. Estos registros se pueden utilizar para almacenar datos, realizar operaciones
aritm\'{e}ticas y l\'{o}gicas, y direccionamiento de memoria.

\subsubsection*{Registro AX}
El registro \textbf{AX} es el registro acumulador de 16 bits. Se utiliza para operaciones de entrada
y salida, y para operaciones aritm\'{e}ticas y l\'{o}gicas.

\subsubsection*{Registro BX}
El registro \textbf{BX} es el registro base de 16 bits. Es el \'{u}nico registro de prop\'{o}sito
general que puede ser utilizado para direccionamiento indexado. Se utiliza para operaciones
aritm\'{e}ticas y l\'{o}gicas.

\subsubsection*{Registro CX}
El registro \textbf{CX} es el registro contador de 16 bits. Se utiliza para contar iteraciones
en ciclos, y para operaciones aritm\'{e}ticas y l\'{o}gicas.

\subsubsection*{Registro DX}
El registro \textbf{DX} es el registro de datos de 16 bits. Se utiliza para operaciones de entrada
y salida, y para operaciones aritm\'{e}ticas y l\'{o}gicas.

\subsection*{Registro apuntadores de pila}
Los registros \textbf{SP} y \textbf{BP} son registros de prop\'{o}sito general de 16 bits. Estos
registros se utilizan para direccionamiento de memoria, y para direccionamiento de pila.

\subsubsection*{Registro SP y BP}
Los registros \textbf{SP} y \textbf{BP} son registros de prop\'{o}sito general de 16 bits. Estos
registros se utilizan para direccionamiento de memoria, y para direccionamiento de pila.

\subsection*{Registro de \textit{banderas}}
El registro \textbf{FLAGS} es un registro de 16 bits, que contiene los \textit{bits} de estado
del procesador. Estos \textit{bits} de estado se utilizan para indicar el resultado de una
operaci\'{o}n aritm\'{e}tica o l\'{o}gica, y para controlar el funcionamiento del procesador.

\begin{flushleft}
    La siguiente tabla muestra los \textit{bits} de estado del registro \textbf{FLAGS}.
\end{flushleft}
\medbreak

\begin{tabular}{|c|c|c|c|c|c|c|c|}
    \hline
    \textbf{Bit} & \textbf{Nombre} & \textbf{Descripci\'{o}n}      & \textbf{Bit} & \textbf{Nombre} & \textbf{Descripci\'{o}n}       \\
    \hline
    0            & CF              & \textit{Carry Flag}           & 8            & OF              & \textit{Overflow Flag}         \\
    \hline
    1            &                 &                               & 9            & DF              & \textit{Direction Flag}        \\
    \hline
    2            & PF              & \textit{Parity Flag}          & 10           & IF              & \textit{Interrupt Enable Flag} \\
    \hline
    3            &                 &                               & 11           & TF              & \textit{Trap Flag}             \\
    \hline
    4            & AF              & \textit{Auxiliary Carry Flag} & 12           & SF              & \textit{Sign Flag}             \\
    \hline
    5            &                 &                               & 13           & ZF              & \textit{Zero Flag}             \\
    \hline
    6            & ZF              & \textit{Zero Flag}            & 14           &                 &                                \\
    \hline
    7            & SF              & \textit{Sign Flag}            & 15           &                 &                                \\
    \hline
\end{tabular}


\end{document}