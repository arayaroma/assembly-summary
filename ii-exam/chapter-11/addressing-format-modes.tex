\documentclass{article}
\usepackage[a4paper, hmargin=1.4in, vmargin=1in]{geometry}
\usepackage{graphicx}

\title{\LARGE\textbf{Repertorios de Instrucciones: \\ 
                     Modos de Direccionamiento y Formatos}}
\author{\large\textit{Daniel Araya Román}}
\date{}

\begin{document}
\maketitle

La referencia a un operando en una instrucci\'{o}n contiene o bien
su valor (inmediato) o una referencia a la direcci\'{o}n del operando.
Diversos repertorios de instrucciones utilizan diferentes modos de
direccionamiento, como lo son, direccionamiento directo, indirecto,
a registro, indirecto con registro, etc.

En el cap\'{i}tulo anterior, se basa en \textit{qu\'{e}} hace una instrucci\'{o}n,
en este cap\'{i}tulo se aborda en \textit{c\'{o}mo} especificar los operandos
y las operaciones de las instrucciones. Teniendo en cuenta, como
\textbf{especificar la direcci\'{o}n de un operando} y
\textbf{como se organizan los bits de una instrucci\'{o}n para definir
las direcciones de los operandos y la operaci\'{o}n que realiza la
instrucci\'{o}n.} \cite{stallings2006organización}

\section*{Modos de Direccionamiento}
Un concepto importante es el \textbf{campo de direcciones}, el cual
en un formato de instrucci\'{o}n, puede estar bastante limitado. Ser\'{i}a
conveniente poder referenciar un rango elevado de posiciones de memoria
principal o virtual. Para este motivo, se utlizan los \textbf{modos de
    direccionamiento}. Implicando compromiso entre el rango de direcciones o
la flexibilidad de direccionamiento. Como tambi\'{e}n, el n\'{u}mero de
referencias a memoria y la complejidad  de c\'{a}lculo de las direcciones.

\subsection*{Direccionamiento Inmediato}
Es la forma m\'{a}s sencilla de direccionamiento, en el que el operando est\'{a}
presente en la propia instrucci\'{o}n.
\begin{verbatim}
                                Operand = A
                                mov ax, 05H
\end{verbatim}

Se puede utilizar para definir constantes, o fijar valores iniciales de variables.
Naturalmente, el n\'{u}mero se almacena en complemento a dos. La \textbf{ventaja}
es que una vez captada la instrucci\'{o}n, no requiere de una referencia a
memoria paraobtener el operando, ahorrando un ciclo de memoria o cach\'{e},
en el ciclo de instrucci\'{o}n. La \textbf{desventaja} es que el tama\~{n}o
del n\'{u}mero est\'{a} restringido a la longitud del
\textbf{campo de direcciones}. Este en la mayor\'{i}a de los repertorios de
instrucciones es peque\~{n}o comparado a una longitud de palabra (16 bits).

\subsection*{Direccionamiento Directo}
Otra forma sencilla de direccionamiento, en el que el campo de direcciones
contiene la direcci\'{o}n efectiva del operando.
\begin{verbatim}
                        EfectiveAddress = A
                        message db 'Hola, mundo!', '$'
                        mov dx, offset message
\end{verbatim}

La \textbf{ventaja} es que solamente requiere una referencia a memoria, la
\textbf{desventaja} es que proporciona un espacio de direcciones limitado.

\subsection*{Direccionamiento Indirecto}
El problema del direccionamiento directo es que la longitud del campo de
direcciones es normalmente menor que la longitud de palabra, limitando el
rango de direcciones. El direccionamiento indirecto permite que el campo de
direcciones referencie la direcci\'{o}n de una palabra de memoria. La cual
contiene la direcci\'{o}n completa del operando.
\begin{verbatim}
                        EfectiveAddress = [A]
                        message db 'Hola, mundo!', '$'
                        mov ax, message
                        mov dx, [ax]
\end{verbatim}

La \textbf{ventaja} obvia es que para una longitud de palabra de \textit{n bits},
se dispone ahora de un espacio de direcciones de \textit{$2^n$}. La
\textbf{desventaja} es que la ejecuci\'{o}n de la instrucci\'{o}n requiere de dos
referencias a memoria, una para obtener la direcci\'{o}n del operando y otra para
obtener el operando en s\'{i}. Una variante del direccionamiento indirecto
poco com\'{u}n es el direccionamiento indirecto multinivel, o en cascada.
\begin{verbatim}
                        EfectiveAddress = [...[A]...]
                        message db 'Hola, mundo!', '$'
                        mov ax, message
                        mov bx, [ax]
                        mov dx, [bx]
\end{verbatim}

No ofrece ninguna ventaja particular, y la desventaja es que requiere de
dos o m\'{a}s referencias a memoria para obtener el operando.

\subsection*{Direccionamiento a Registro}
Es an\'{a}logo al direccionamiento directo, la diferencia es que el campo de
direcciones referencia un registro en lugar de una direcci\'{o}n de memoria
principal.

\begin{verbatim}
                        EfectiveAddress = R
                        .model small
                        .data
                            value1 dw 10
                            value2 dw 20
                            result dw ?

                        .code
                        main proc
                            mov ax, value1
                            add ax, value2
                            mov result, ax

                            mov ah, 4Ch
                            int 21h
                        main endp
                        end main
\end{verbatim}

Las \textbf{ventajas} son que solo es necesario un campo peque\~{n}o de
direcciones en la instrucci\'{o}n, y que no se requieren referencias a memoria.
Siendo el tiempo de acceso a un registro interno es mucho menor que para la
memoria principal. Las \textbf{desventajas} es que el espacio de direcciones
es limitado y que el n\'{u}mero de registros es limitado.

Si se hace un uso masivo del direccionamiento a registros en un repertorio de
instrucciones, los registros del procesador se emplear\'{a}n intensamente.
Depende del programador decidir qu\'{e} valores deben mantenerse en registros
y cu\'{a}les deben almacenarse en memoria principal. La mayor\'{i}a de los
procesadores modernos emplean m\'{u}ltiples registros de uso general, cargando
al programador en lenguaje ensamblador (cuando se desarrolla compiladores,
por ejemplo) con la responsabilidad de conseguir una ejecuci\'{o}n eficiente.

\subsection*{Direccionamiento Indirecto a Registro}
Igual que el direccionamiento de registro es an\'{a}logo al direccionamiento
directo, el direccionamiento indirecto a registro es an\'{a}logo al
direccionamiento indirecto. La diferencia estriba en si el campo de direcciones
hace referencia a una posici\'{o}n de memoria principal o a un registro.

\begin{verbatim}
                        EfectiveAddress = [R]
                        .model small
                        .data
                            value       dw 10
                            value_ptr   dw offset value
                            result      dw ?

                        .code
                        main proc
                            mov ax, value_ptr
                            mov ax, [ax]
                            add ax, 5
                            mov result, ax

                            mov ah, 4Ch
                            int 21h
                        main endp
                        end main
\end{verbatim}

Las \textbf{ventajas} y limitaciones son las mismas que para el direccionamiento
indirecto. Em ambos casos la limitaci\'{o}n es que el campo de direcciones,
se supera haciendo que dicho campo referencie una posici\'{o}n de memoria
que contenga la direcci\'{o}n del operando. Adem\'{a}s, el direccionamiento
indirecto a registro requiere de una referencia menos a memoria para
obtener el operando.

\subsection*{Direccionamiento con Desplazamiento}
Un modo de direccionamiento que combina las posibilidades de los direccionamientos
directo e indirecto con registro.
\begin{verbatim}
                        EfectiveAddress = A + (R)
                        .model small
                        .data
                            array   dw 1, 2, 3, 4, 5
                            index   dw 2
                            result  dw ?

                        .code
                        main proc
                            mov ax, index
                            mov bx, offset array

                            mov cx, 2
                            mov ax, [bx + ax * cx]
                            mov result, ax

                            mov ah, 4Ch
                            int 21h
                        main endp
                        end main
    
\end{verbatim}

Requiere que tenga dos campos de direcciones, al menos uno de los cuales
es expl\'{i}cito. El valor contenido en uno de los campos de direcciones
se utiliza directamente (valor = A), el otro campo de direcciones, o una
referencia impl\'{i}cita, se refiere a un registro (valor = R), cuyo contenido
se suma a A para obtener la direcci\'{o}n efectiva del operando. Los tres usos
m\'{a}s comunes de este modo de direccionamiento son:

\begin{itemize}
    \item \textbf{Desplazamiento base.}
          El registro referenciado contiene una direcci\'{o}n de memoria, y
          el campo de direcci\'{o}n contiene un desplazamiento desde esa
          direcci\'{o}n base. Siendo la referencia expl\'{i}cita o impl\'{i}cita.
    \item \textbf{Desplazamiento indexado.}
          El campo de direcci\'{o}n referencia una direcci\'{o}n de memoria,
          y el registro referenciado contiene un desplazamiento positivo desde
          esa direcci\'{o}n. Obs\'{e}rvese que es justo lo opuesto de la
          interpretaci\'{o}n del desplazamiento base. Puede ser preindexado o
          postindexado.
    \item \textbf{Desplazamiento relativo.}
          Tambi\'{e}n llamado direccionamiento relativo al PC, el registro
          referenciado impl\'{i}citamente es el contador de programa (PC).
          La direcci\'{o}n de la instrucci\'{o}n actual se suma al campo de
          direcciones para obtener la direcci\'{o}n efectiva del operando.
\end{itemize}

\subsection*{Direccionamiento de Pila}
La pila tiene un puntero asociado, cuya valor es la direcci\'{o}n de la cabecera
o tope de la pila. Este puntero se mantiene en un registro, el cual se denomina
\textbf{Stack Pointer (SP)}. El direccionamiento de pila es un caso especial,
porque las referencias a posiciones de la pila en memoria son, de hecho,
direcciones de acceso indirecto con registro. El modo de direccionamiento de
pila es una forma de direccionamiento impl\'{i}cito. Las instrucciones m\'{a}quina
no necesitan incluir la referencia a memoria, sino que se operan con la cabecera
de la pila.

\section*{Modos de direccionamiento del Pentium}
El mecanismo de traducci\'{o}n de direcciones del Pentium produce una direcci\'{o}n
denominada virtual o efectiva, que es un desplazamiento dentro de un segmento.
La suma de la direcci\'{o}n de comienzo del segmento y la direcci\'{o}n efectiva
produce una \textbf{direcci\'{o}n lineal.}

\begin{figure}[h]
    \centering
    \includegraphics[width=4.5in]{pentium_addressing_modes.png}
    \caption{Modos de direccionamiento del Pentium
        \cite{stallings2006organización}}
    \label{fig:modos-direccionamiento-pentium}
\end{figure}

El pentium contiene seis registros de segmento, cada uno de los cuales contiene
la direcci\'{o}n de comienzo de un segmento. Adem\'{a}s de dos registros que
pueden emplearse para construir una direcci\'{o}n, el \textbf{registro base (BX)}
y el \textbf{registro \'{i}ndice (SI)}.

\begin{figure}[h]
    \centering
    \includegraphics[scale=0.47]{pentium_addressing_modes2.png}
    \caption{Modos de direccionamiento del Pentium
        \cite{stallings2006organización}}
    \label{fig:modos-direccionamiento-powerpc}
\end{figure}

\section*{Formatos de Instrucciones}
Un formato de instrucciones define la descripci\'{o}n en bits de una instrucci\'{o}n.
Debe de incluir un \textbf{c\'{o}digo de operaci\'{o}n (codop)}, e incluir
cero o m\'{a}s operandos. Cada operando expl\'{i}cito se referencia utlizando
uno de los \textbf{modos de direccionamiento}. En la mayor\'{i}a de los repertorios
de instrucciones se emplean m\'{a}s de un formato de instrucci\'{o}n.

\subsection*{Longitud de Instrucci\'{o}n}
El aspecto m\'{a}s b\'{a}sico a considerar es la longitud de la instrucci\'{o}n.
Afecta y se ve afectada por el tama\~{n}o de la memoria, su organizaci\'{o}n,
la estructura de buses, la complejidad del procesador y la velocidad del mismo.
Esta decisi\'{o}n define la riqueza y la flexibilidad de la m\'{a}quina desde un
punto de vista del programador en lenguaje ensamblador.

El programador desea m\'{a}s \textit{codops}, m\'{a}s operandos, m\'{a}s modos de
direccionamiento y mayor rango de direcciones. Todo esto requiere de bits, y
empuja hacia longitudes de instrucci\'{o}n mayores. No obstante una instrucci\'{o}n
mayor puede ser improcedente. \textbf{No porque una instrucci\'{o}n de 64 bits ocupe el
doble de espacio que una de 32 bits, significa que sea el doble de \'{u}til.}
La longitud de palabra de memoria, \textbf{(16 bits)}, es, en cierto sentido, la unidad natural de
organizaci\'{o}n.

\subsection*{Asignaci\'{o}n de los Bits}
Un aspecto igualmente dif\'{i}cil es c\'{o}mo asignar los bits en un formato de
instrucci\'{o}n. Para una longitud de instrucci\'{o}n dada, existe un compromiso
entre el n\'{u}mero de \textit{codops} y la capacidad de direccionamiento. Un
mayor n\'{u}mero de \textit{codops} implica m\'{a}s bits en el campo de codop.
Reduciendo el n\'{u}mero de bits disponibles para direccionamiento. Los siguientes
factores influyen en la asignaci\'{o}n de bits:

\begin{itemize}
    \item \textbf{N\'{u}mero de modos de direccionamiento.}
          Pueden a veces indicarse de forma impl\'{i}cita. En otros casos,
          deben de especificarse expl\'{i}citamente. requiriendo uno o m\'{a}s
          bits de modo.
    \item \textbf{N\'{u}mero de operandos.}
          Las instrucciones t\'{i}picas de las m\'{a}quinas actuales permiten
          dos operandos. Cada direcci\'{o}n de operando podr\'{i}a requerir de un
          indicador de modo dentro de la instrucci\'{o}n.
    \item \textbf{Registros frente a memoria.}
          Una m\'{a}quina debe de disponer de registros para traer datos al
          procesador a fin de procesarlos. En el caso de solo tener un registro
          visible (el acumulador), la direcci\'{o}n del operando est\'{a}
          impl\'{i}cita y no consume bits. Incluso con varios registros, se ocupan
          pocos bits para especificar el registro. Normalmente se disponen de
          8 a 32 registros visibles para el usuario.
    \item \textbf{N\'{u}mero de conjuntos de registros.}
          Distintas m\'{a}quinas tienen un conjunto de registros de uso general,
          entre 8 y 16 registros. Una ventaja que ofrece este enfoque es que,
          requiere menos bits por instrucci\'{o}n. As\'{i} el codop determina
          de forma impl\'{i}cita qu\'{e} conjunto de registros se est\'{a}
          referenciando.
    \item \textbf{Rango de direcciones.}
          Para referencias a memoria, este rango est\'{a} relacionado con el
          n\'{u}mero de bits disponibles para direccionamiento. Dado que esto
          impone limitaciones, rara vez se utiliza el direccionamiento directo,
          normalmente se emplea el direccionamiento con desplazamiento.
    \item \textbf{Granularidad de las direcciones.}
          Para direcciones referenciando a memoria en lugar de registros, otro factor
          que afecta es su granularidad de direccionamiento. El direccionamiento por
          bytes es conveniente manipular caracteres pero requiere, para un tama\~{n}o
          de memoria dado, m\'{a}s bits de direcci\'{o}n.
\end{itemize}

\subsection*{PDP-8}
Un dise\~{n}o sencillo para un computador de uso general fue el del PDP-8.
Este utiliza instrucciones de 12 bits, opera con palabras de 12 bits, y solo
tiene un registro de uso general, el acumulador. El formato de instrucci\'{o}n
del PDP-8 es bastante eficiente. Permite direccionamiento indirecto, direccionamiento
con desplazamiento, e indexado. Dispone de un total de 35 instrucciones.

\begin{figure}[h]
    \centering
    \includegraphics[scale=0.4]{pdp-8.png}
    \caption{Formato de instrucciones del PDP-8
        \cite{stallings2006organización}}
    \label{fig:formato-instruccion-pdp8}
\end{figure}

\subsection*{PDP-10}
Tiene una longitud de instrucci\'{o}n de 36 bits, y una longitud de palabra de
36 bits. Permite 512 operaciones, con 365 instrucciones diferentes. Con 16
registros de uso general.

\begin{figure}[h]
    \centering
    \includegraphics[scale=0.4]{pdp-10.png}
    \caption{Formato de instrucci\'{o}n del PDP-10
    \cite{stallings2006organización}}
    \label{fig:formato-instruccion-pdp10}
\end{figure}


\subsection*{Instrucciones de Longitud Variable: PDP-11}
El PDP-11 se dise\~{n}\'{o} para proporcinar un repertorio de instrucciones
flexible dentro de las limitaciones de un microprograma de 16 bits. Este emplea
8 registros de uso general de 16 bits, teniendo punteros de pila. Una ventaja
que posee es el modo de direccionamiento con operando, en lugar de con el codop.
Esta independencia se denomina \textit{ortogonalidad}. Tiene 13 formatos de
instrucci\'{o}n diferentes, con una longitud de instrucci\'{o}n de 16 bits.

\begin{figure}[h]
    \centering
    \includegraphics[scale=0.38]{pdp-11.png}
    \caption{Formatos de instrucciones del PDP-11
        \cite{stallings2006organización}}
    \label{fig:formato-instruccion-pdp11}
\end{figure}

\newpage
\subsection*{Formatos de instrucci\'{o}n del Pentium}

\begin{figure}[h]
    \centering
    \includegraphics[scale=0.4]{pentium_instruction_format.png}
    \caption{Formatos de instrucci\'{o}n del Pentium
    \cite{stallings2006organización}}
    \label{fig:formato-instruccion-pentium}
\end{figure}


\bibliography{stallings}
\bibliographystyle{plain}

\begin{figure}[ht]
    \centering
    \includegraphics[width=4in]{addresing_modes.png}
    \caption{Modos de direccionamiento
        \cite{stallings2006organización}}
    \label{fig:modos-direccionamiento}
\end{figure}

\end{document}